% apps/web/server/bilan/templates/bilan_report.tex
\documentclass[11pt]{article}
\usepackage[margin=20mm]{geometry}
\usepackage{fontspec}
\usepackage{xcolor}
\usepackage{hyperref}
\usepackage{hyperxmp}
\usepackage{longtable}
\usepackage{array}
\usepackage{microtype}
\usepackage{amsmath, amssymb, amsthm}
\usepackage[french]{babel}
\frenchspacing
% Font configured in includes/fonts.tex
% Includes (colors/fonts/charts)
% includes/colors.tex
\definecolor{NexusBlue}{HTML}{2563EB}
\definecolor{NexusViolet}{HTML}{7C3AED}
\definecolor{NexusSlate}{HTML}{0F172A}
\definecolor{NexusGreen}{HTML}{10B981}
\definecolor{NexusAmber}{HTML}{F59E0B}
\definecolor{NexusRed}{HTML}{EF4444}


% includes/fonts.tex
% Requires: \usepackage{fontspec}
% Ensure XeLaTeX or LuaLaTeX
\setmainfont{Inter}[
  Path=fonts/,
  UprightFont = *-Regular,
  ItalicFont  = *-Italic,
  BoldFont    = *-Bold,
  BoldItalicFont = *-BoldItalic
]
\setsansfont{Inter}[
  Path=fonts/,
  UprightFont = *-Regular,
  ItalicFont  = *-Italic,
  BoldFont    = *-Bold,
  BoldItalicFont = *-BoldItalic
]
\newfontfamily\nexusheading{Inter}[
  Path=fonts/,
  UprightFont = *-SemiBold
]


% includes/charts.tex
% Requires: \usepackage{pgfplots}, \usepackage{tikz}
\pgfplotsset{compat=1.18}
\definecolor{NexusTeal}{RGB}{12,155,140}

% Draw a radar chart from precomputed polar coordinates.
% Usage: \RadarFromCoords{ (0,80) (60,65) (120,50) (180,72) (240,90) (300,55) }
\newcommand{\RadarFromCoords}[2][]{%
  % #1 extra options, #2 coords string
  \begin{tikzpicture}
    \begin{polaraxis}[
      grid=both,
      xticklabel style={align=center, text width=2.2cm, font=\small},
      ymin=0, ymax=100,
      ytick={0,25,50,75,100},
      xticklabel style={font=\scriptsize},
      yticklabel style={font=\scriptsize},
      axis lines=none,
      #1
    ]
      \addplot+[mark=*, color=NexusBlue, fill=NexusBlue!18] coordinates { #2 } -- cycle;
    \end{polaraxis}
  \end{tikzpicture}%
}

% Draw bars from symbolic x coords and (label,value) coordinates.
% Usage: \BarsFromCoords{Label A,Label B,Label C}{ (Label A,80) (Label B,65) (Label C,50) }
\newcommand{\BarsFromCoords}[3][]{%
  % #1 extra options, #2 labels CSV, #3 coords
  \begin{tikzpicture}
    \begin{axis}[
      ybar,
      bar width=8pt,
      ymin=0, ymax=100,
      xtick=data,
      symbolic x coords={#2},
      axis x line*=bottom,
      axis y line*=left,
      enlarge x limits=0.2,
      ylabel={Score},
      tick label style={font=\scriptsize},
      #1
    ]
      \addplot+[draw=NexusBlue, fill=NexusBlue!25] coordinates{ #3 };
    \end{axis}
  \end{tikzpicture}%
}

% Timeline bars from precomputed coordinates per phase
% Usage: \NxsTimelineBars{ (S1,1) (S2,0) ... }{ ... }{ ... }
\newcommand{\NxsTimelineBars}[3]{%
  \begin{tikzpicture}
    \begin{axis}[
      width=\linewidth, height=5.2cm,
      ybar, bar width=0.55cm,
      ymin=0, ymax=1,
      enlarge x limits=0.06,
      symbolic x coords={S1,S2,S3,S4,S5,S6,S7,S8},
      xtick=data,
      ytick={0,0.25,0.5,0.75,1}, yticklabels={0,25,50,75,100},
      grid=both, minor y tick num=1,
      ylabel={Intensité}, xlabel={Semaines},
      legend style={draw=none, at={(0.5,1.05)}, anchor=south, legend columns=-1},
      axis line style={draw=none},
      tick style={draw=none},
    ]
      \addplot+[draw=none, fill=NexusGreen] coordinates { #1 };
      \addlegendentry{Consolidation}
      \addplot+[draw=none, fill=NexusBlue] coordinates { #2 };
      \addlegendentry{Approfondissement}
      \addplot+[draw=none, fill=NexusAmber] coordinates { #3 };
      \addlegendentry{Entraînement}
    \end{axis}
  \end{tikzpicture}%
}


\hypersetup{
  pdftitle={Nexus Réussite — Rapport de Bilan},
  pdfauthor={ARIA • Nexus Réussite},
  pdfsubject={Bilan académique et pédagogique premium},
  pdfkeywords={Nexus, Bilan, NSI, Mathématiques, Parcoursup},
  colorlinks=true, linkcolor=NexusBlue, urlcolor=NexusBlue
}
\usepackage{fancyhdr}
\pagestyle{fancy}
\fancyhf{}
\lhead{\textsf{Nexus Réussite}}
\rhead{\textsf{\thepage}}
% Sections compatibles avec la table des matières
\newcommand{\Section}[1]{%
  \section*{#1}\addcontentsline{toc}{section}{#1}\vspace{-6pt}%
}

\begin{document}

% Page de garde
{\centering
\vspace*{10mm}
{\nexusheading\Large\color{NexusSlate} Nexus Réussite}\\[2mm]
{\Huge\bfseries\color{NexusBlue} Bilan Premium}\\[4mm]
{\large\color{NexusSlate} {{meta.matiere}} — {{meta.niveau}} — {{meta.statut}}}\\[2mm]
{\small Généré le: {{meta.createdAtISO}}}\\
\par}
\vspace{10mm}
\hrule height 2pt \color{NexusBlue}
\vspace{8mm}
\noindent\textbf{Élève:} {{eleve.firstName}} {{eleve.lastName}}\hfill\textbf{Variant:} {{meta.variant}}
\vspace{10mm}

% Table des matières
\setcounter{tocdepth}{2}
\tableofcontents
\clearpage

\Section{Profil}
\begin{itemize}
  \item Matière: {{meta.matiere}} — Niveau: {{meta.niveau}} — Statut: {{meta.statut}}
  \item Etablissement: {{eleve.etab}}
\end{itemize}

\Section{Forces}
\begin{itemize}
{{#forces}}\item {{.}}{{/forces}}
{{^forces}}\item (Aucune){{/forces}}
\end{itemize}

\Section{Axes de progression}
\begin{itemize}
{{#faiblesses}}\item {{.}}{{/faiblesses}}
{{^faiblesses}}\item (Aucun){{/faiblesses}}
\end{itemize}

\Section{Lacunes critiques}
\begin{itemize}
{{#lacunes}}\item {{.}}{{/lacunes}}
{{^lacunes}}\item (Aucune){{/lacunes}}
\end{itemize}

\Section{Scores par domaine}
\begin{longtable}{p{60mm} p{25mm}}
\textbf{Domaine} & \textbf{Maîtrise}\\\hline
{{#diagnostic_rows}}
{{label}} & {{percent}}\%\\
{{/diagnostic_rows}}
\end{longtable}
\vspace{2mm}
\noindent\textbf{Score global:} {{qcm.scoreGlobal}}\%\\[4mm]
\noindent\textbf{Radar des compétences}\\par
\RadarFromCoords{ {{radar.coords}} }\\[2mm]
{\small\textit{Axes:} {{#radar.labels}}{{.}}{{^@last}} • {{/@last}}{{/radar.labels}} }\\[4mm]
\noindent\textbf{Répartition par domaines}\par
\BarsFromCoords{ {{bars.labelsCsv}} }{ {{bars.coords}} }

\Section{Profil pédagogique}
\begin{itemize}
  \item Style: {{profil.style}} \quad Autonomie: {{profil.autonomie}}
  \item Organisation: {{profil.organisation}} \quad Stress: {{profil.stress}}
  \item Flags: {{#profil.flags}}{{.}}, {{/profil.flags}}
\end{itemize}

\Section{Feuille de route}
Horizion: {{plan.horizonMois}} mois — Charge hebdo: {{plan.hebdoHeures}} h\\
\begin{longtable}{p{18mm} p{140mm}}
\textbf{\#} & \textbf{Étape}\\\hline
{{#plan_rows}} {{idx}} & {{text}}\\ {{/plan_rows}}
\end{longtable}
\vspace{2mm}
\Section{Feuille de route (8 semaines)}
\NxsTimelineBars{ {{timeline.c}} }{ {{timeline.d}} }{ {{timeline.t}} }
\smallskip
\textbf{Détail par semaine :}\\
\begin{itemize}
  \item \textbf{S1} — {{TLdOne}}
  \item \textbf{S2} — {{TLdTwo}}
  \item \textbf{S3} — {{TLdThree}}
  \item \textbf{S4} — {{TLdFour}}
  \item \textbf{S5} — {{TLdFive}}
  \item \textbf{S6} — {{TLdSix}}
  \item \textbf{S7} — {{TLdSeven}}
  \item \textbf{S8} — {{TLdEight}}
\end{itemize}

\Section{Offre Nexus recommandée}
\textbf{Principale:} {{offre.primary}}\\
\textbf{Alternatives:} {{#offre.alternatives}}{{.}}, {{/offre.alternatives}}\\
\textbf{Justification:} {{offre.reasoning}}

\Section{Références et extraits}
\begin{enumerate}
{{#citations}}
  \item \textbf{ {{title}} } — \url{ {{src}} }\\ \emph{«\,{{snippet}}\,»}
{{/citations}}
{{^citations}}
  \item (Aucune citation)
{{/citations}}
\end{enumerate}

\end{document}
